\chapter{Introduction}
\label{Introduction}

	Ce projet a pour but d'étudier la factorisation d'un polynôme en facteurs irréductibles sur des corps finis à l'aide d'algorithmes de factorisation.

\section{Historique}
\label{Historique}

	La théorie des corps finis, notamment l'étude des congruences, sur des entiers et sur des polynômes, se développe à partir du 19ième siècle.
	Puis en 1967, Elwyn Berlekamp développa le premier algorithme de factorisation de polynômes sur un corps.
	Tandis qu'en 1981 avec l'algorithme de Cantor-Zassenhaus, on a l'apparition d'algorithme plus performant.

\section{Quelques propriétés fondamentales d'un corps fini}
\label{Quelques propriétés fondamentales d'un corps fini}

\newtheorem*{def1}{Définition 1.2.1}
\begin{def1}[Polynôme irréductible]
	Soient $K$ un corps et $P\in K[X]$. On dit que $P$ est irréductible (dans $K[X]$) si $P$ n’est pas constant et si ses seuls diviseurs sont les polynômes constants et les polynômes associés à $P$, ie de la forme $\lambda P$ avec $\lambda\in K^{\ast}$.
\end{def1}

\newtheorem*{theo1}{Théorème 1.2.1}
\begin{theo1}[Wedderburn]
	Tout corps fini est commutatif.
\end{theo1}

\begin{proof}[Preuve]
	Soit $K$ un corps fini non nécessairement commutatif de caractéristique $p>0$.
\newline
	On pose $Z=Z(K)=\left\{x\in K\mid\forall y\in K,xy=yx\right\}$ le centre de $K$.
	Il est clair que $Z$ est un sous-corps fini commutatif de $K$, donc de caractéristique $p$. 
	Par conséquent, $K$ est le sous-corps premier de $\mathbb{F}_{p}$.
	Ainsi on a que $Z$ est un $\mathbb{F}_{p}$-espace vectoriel de dimension $r$ et $K$ un $Z$-espace vectoriel de dimension $n$.
	Si on défini $q=card(Z)$, on a alors que $q=p^{r}$ et $card(K)=q^{n}$.

	Soit $x\in K^{*}$, on note $C_{x}=\left\{y\in K\mid xy=yx\right\}$ le centralisateur de $x$ dans $K$.
	On a clairement que $C_{x}\supset Z$ est un sous-corps de $K$, alors $\exists d(x)\in N^{*}$ tel que $card(C_{x})=q^{d(x)}$.
	Comme $C^{*}_{x}$ est un sous-groupe de $K^{*}$, on que $card(C^{*}_{x})\mid card(K^{*})$, ie. $q^{d(x)}-1\mid q^{n}-1$ 

	On veut ainsi montrer que $d(x)=n=1$.
	On applique la division euclidienne de $n$ par $d(x)$,ie. $n=s(x)d(x)+t(x)$. Il en suit : 
	\begin{align*}
	q^{n}-1 &= \left((q^{d(x)})^{s(x)}-1\right)q^{t(x)}+q^{t(x)}-1\\
					&= (q^{d(x)}-1)\left(\sum^{s(x)-1}_{i=0}q^{id(x)+t(x)}\right)+q^{t(x)}-1.
	\end{align*}
	Par conséquent, $q^{d(x)}-1\mid q^{t(x)}-1$. Comme $q\geq 2$ et $0\leq t(x)<d(x)$, d'où $t(x)=0$, donc on a $d(x)\mid n$.

	Soit $n>1$, on suppose par l'absurde que $K$ est non commutatif.
	On considère l'action du groupe $K^{*}$ sur lui-même par conjugaison; on obtient donc une partition de $K^{*}$ telle que $\mathcal{P}(K^{*})=\bigcup_{x_{i}\in K^{*}}\vartheta_{x_{i}}$ l'union disjointe finie d'orbites.
	Ceci nous donne l'équation aux classes suivante : 
	\begin{align}
	q^{n}-1 &= q-1+\sum_{x\in\mathcal{P}(K^{*}) et x\notin Z}\frac{q^{n}-1}{q^{d(x)}-1}.\\
	\intertext{Soit $\phi_{m}$ le polynôme cyclotomique, on a d'après les propriétés de $\phi_{m}$ que :}
	q^{n}-1 &= \prod_{m\mid n}\phi_{m}(q),\\
	q^{d}-1 &= \prod_{m\mid d}\phi_{m}(q)
	\end{align}
	On remarque que si $d\mid n$ et $d<n$, on a $\phi_{n}(q)\mid\frac{q^{n}-1 }{q^{d}-1}$.
	Alors, d'après $\eqref{}$, $\phi_{n}(q)\mid q-1$, donc $\left|\phi_{n}(q)\right|\leq q-1$.

	Soit $\xi\in U^{*}_{n}{(\mathbb{C})}$. Comme $n>1$, on a $\xi\neq 1$, d'où $\left|q-\xi\right|>q-1$.
	On en déduit que $\phi_{n}(q)$ ne divise pas $q-1$ dans $\mathbb{Z}$, d'où la contradiction.
	Par conséquent, $n=1$, $K=Z$ et $K$ est commutatif.
\end{proof}

\newtheorem*{prop1}{Proposition 1.2.1}
\begin{prop1}
	Soient $K$ un corps fini caractéristique $p>0$ un nombre premier $p>0$. On a :
	\begin{enumerate}
		\item $Card(K)=q=p^{n}$ avec un entier $n\geq1$.
		\item Le groupe $(K^{*},\times)$ est cyclique d'ordre $q-1$.
		\item $\forall x\in K^{*}$, $x^{q-1}=1$ et $\forall x\in K$, $x^{q}=x$
	\end{enumerate}
\end{prop1}

\newtheorem*{prop2}{Proposition 1.2.2}
\begin{prop2}
	Le groupe $(K,+)$ est isomorphe à $(\mathbb{Z}/p\mathbb{Z})^{n}$.
\end{prop2}
%\begin{proof}[Preuve]
%Soient A un anneau et 1A son élément unité. Si 1A est d’ordre fini p ∈ N∗ dans le groupe additif
%(A, +), on dit que A est de caractéristique p. Dans ce cas p est le plus petit entier non nul
%vérifiant k1A = 0. Si par contre 1A n’est pas d’ordre fini, alors k1A = 0 pour tout k ∈ N∗
%et
%l’on dit que A est un anneau de caractéristique nulle. L’application
%\end{proof}
