\chapter{Conclusion}
\label{Conclusion}

	Nous avons pu voir que la factorisation d'un polynôme en facteurs irréductibles est assez simple pour $p$ assez petit grâce à l'algorithme de Berlekamp, mais cette recherche de factorisation devient assez vite fastidieuse pour $p$ grand. Ceci est dû à sa complexité polynomiale.
	Par conséquent pour faire face à ce problème, on utilise d'autres algorithmes probabilistes avec toujours une complexité polynomiale, cependant plus rapide que la méthode d'algèbre linéaire.
	Malheureusement, on ne connaît pas encore d'algorithme déterministe avec une complexité linéaire ou quasi-linéaire dans des corps finis.\\

	Grâce à cette factorisation dans $\mathbb{F}_{p}$, on peux trouver une factorisation dans $Z$ à l'aide d'algorithme de remontée de Hensel linéaire ou quadratique.
	On peux retrouver une autre utilisation de la factorisation dans des corps fini de polynômes dans les systèmes de calculs formels, en cryptographie notamment dans le calcul de logarithme discret, \ldots