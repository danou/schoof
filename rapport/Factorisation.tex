\chapter{Algorithme de factorisation}
\label{Algorithme de factorisation}

	Dans la littérature, il existe deux algorithmes majeurs de factorisation de polynômes en facteurs irréductibles sur un corps fini.

\section{Réduction au cas sans facteur carré}
\label{Réduction au cas sans facteur carré}

	Afin de pouvoir factoriser un polynôme quelconque, on doit le décomposer en facteur carrés avant de pourvoir utiliser un algorithme de factorisation.
	On pose $P=P_{1}^{\alpha_{1}}\ldots P_{s}^{\alpha_{s}}\in\mathbb{F}_{p}[X]$ polynôme quelconque où les $P_{i}$ sont irréductibles (deux à deux distincts) et les $\alpha_{i}\geq1$.

\newtheorem*{def2}{Définition 4.1.1}
\begin{def2}
	Soit $P\in\mathbb{F}_{p}[X]$ un polynôme.
	On dit que $P$ est sans carré s’il n’existe pas de polynôme non constant $Q\in\mathbb{F}_{p}[X]$ tel que $Q^{2}\mid P$.
\end{def2}

\newtheorem*{lem1}{Lemme 4.1.1}
\begin{lem1}
	Soit $P\in\mathbb{F}_{p}[X]$.
	La dérivée $P'$ vaut 0 si et seulement si $P$ s'écrit sous la forme $Q^{p}$, $Q\in\mathbb{F}_{p}[X]$.
\end{lem1}

\begin{proof}[Preuve]
	Si $P=Q^{p}$, on a bien que $P'=Q^{p-1}pQ'=0$.

	Réciproquement, si $P'=0$, $P=\sum_{i}{q_{i}X^{pi}}=\sum_{i}{q_{i}^{p}X^{pi}}=\sum_{i}{(q_{i}X^{i})^{p}}=\left(\sum_{i}{q_{i}X^{pi}}\right)^{p}=Q^{p}$.
\end{proof}

\newtheorem*{prop7}{Proposition 4.1.1}
\begin{prop7}
	Soit $P\in\mathbb{F}_{p}[X]$.
	$P$ est sans facteur carré si et seulement si $pgcd(P,P')=1$ avec $P'$ la dérivée de $P$.
\end{prop7}

\begin{proof}[Preuve]
	Soit $P=P_{1}^{\alpha_{1}}\ldots P_{s}^{\alpha_{s}}\in\mathbb{F}_{p}[X]$ où les $P_{i}$ sont irréductible et les $\alpha_{i}\geq1$.
	On obtient en dérivant $P$ : $$P'=\sum_{i=1}^{s}{\alpha_{i}P'_{i}\frac{P}{P_{i}}.$$

	Si on $P$ admet un facteur carré, il existe au moins un $\alpha_{i}\geq2$ où $P_{i}\mid P'_{i}\frac{P}{P_{i}}$.
	Alors $P_{i}\mid P'_{i}\frac{P}{P_{j}},\ \forall i\neq j$, il divise $P'$ de sorte que $pgcd(P,P')\neq1$.

	Réciproquement, comme les $P_{i}$ sont irréductibles, d'après le lemme 4.1.1 alors on sait que leur dérivées sont non nulles.
	On suppose donc que $pgcd(P,P')\neq1$. Par conséquent, il existe un $P_{i}\mid P'$.
	En effet, il divise tout les autres $\frac{P}{P_{j}}$, et ainsi divise $\alpha_{i}P'_{i}\frac{P}{P_{i}}$.
	Ainsi $P'_{i}$ est non nul, donc $P_{i}\mid\alpha_{i}P_{i}\frac{P}{P_{i}}$.

	Si $\alpha_{i}\not\equiv0\ [p]$, on a $P_{i}\mid P'_{i}\frac{P}{P_{i}}$ et $pgcd(P,P')=1$, donc, d'après le lemme de Gauss, finalement $P_{i}^{2}\mid P$.

	Sinon si $\alpha_{i}\equiv0\ [p]$, on a $p\mid\alpha_{i}$, donc finalement $\alpha_{i}\neq1$.
\end{proof}

Maintenant, on récrit $P=Q_{1}Q_{2}^{2}\ldots Q_{s}^{s}$ tel que $Q_{1}$ le produit des $P_{i}$ apparaissant à la puissance $1$, \ldots, $Q_{s}$ le produit des $P_{i}$ apparaissant à la puissance $s$. On appelle alors $Q_{1}Q_{2}\ldots Q_{s}$ la partie sans facteur carré de $P$.

\newtheorem*{prop8}{Proposition 4.1.2}
\begin{prop8}
	Si $p>s$, alors la partie sans carré de $P$ est $P/pgcd(P,P')$.
\end{prop8}

\begin{proof}[Preuve]
	On obtient en dérivant $P=Q_{1}Q_{2}^{2}\ldots Q_{s}^{s}$ : $$P'=(Q_{2}\ldots Q_{s}^{s-1})(Q'_{1}Q_{2}^{2}\ldots Q_{s}^{s}+\ldots+sQ_{1}Q_{2}^{2}\ldots Q_{s-1}').$$
	Les $Q'_{i}$ ne sont pas nuls d'après le lemme 4.1.1.
	Comme $\forall1\leq i\leq s,\ i\not\equiv0\ [p]$, le membre de droite est donc premier avec tous les $Q_{i}$, donc avec $P$.
	Donc $pgcd(P,P')=Q_{2}\ldots Q_{s}^{s-1}$.
\end{proof}

\newtheorem*{prop9}{Proposition 4.1.3}
\begin{prop9}
	On a $$\frac{P}{pgcd(P,P')}=\prod\limits_{\substack{i\ non\ multiple\ de\ p}}{Q_{i}^{i}}.$$
\end{prop9}

\begin{proof}[Preuve]
	Comme la preuve précédente, on remarque que $$pgcd(P,P')=(Q_{2}\ldots Q_{s}^{s-1})\cdot\prod\limits_{\substack{i\ multiple\ de\ p}}{Q_{i}^{i}}.$$
\end{proof}

\newtheorem*{prop10}{Proposition 4.1.4}
\begin{prop10}
	Soient $n=deg(P)$, $U=pgcd(P,P')$ et $V=P/U$.

	On pose $W=pgcd(U,V^{n})$. Alors $$U/W=\prod\limits_{\substack{i\ multiple\ de\ p}}{Q_{i}^{i}}.$$
\end{prop10}

\begin{proof}[Preuve]
	D'après la proposition précédente, $$V^{n}=\prod\limits_{\substack{i\ non\ multiple\ de\ p}}{Q_{i}^{n}$$ et $$U=(Q_{2}\ldots Q_{s}^{s-1})\cdot\prod\limits_{\substack{i\ multiple\ de\ p}}{Q_{i}}.$$
	Comme $n$ est la plus grande multiplicité, on a par conséquent que $$W=\prod\limits_{\substack{i\ non\ multiple\ de\ p}}{Q_{i}^{i-1}.$$
\end{proof}


	Voici donc l'algorithme de décomposition sans facteur carré :

\begin{algorithm}
\caption{Calculer la partie sans facteur carré d'un polynôme}
\begin{algorithmic} 
\REQUIRE $P\in\mathbb{F}_{p}[X]$ un polynôme quelconque.
\ENSURE La partie sans facteur carré de $P$.
\STATE $U=pgcd(P,P')$;
\STATE $V=P/U$;
\STATE $W=pgcd(U,V^{n})$;
\STATE Calculer la racine p-ième $W_{0}$ de $W$;
\STATE Calculer récursivement la partie sans facteur carré $S$ de $W_{0}$;
\RETURN $VS$.
\end{algorithmic}
\end{algorithm}

\newtheorem*{prop13}{Proposition 4.1.5}
\begin{prop13}
	L’algorithme a une complexité en $O(M(n) log(n))$ où $M$ est une fonction de multiplication.
\end{prop13}

\begin{proof}
	Les calculs de $pgcd$ se font en $O(M(n) log(n))$ opérations ; remarquer qu'il suffit de calculer $V^{n}\ mod\ U$, et que cela se fait en $O(log(n))$ multiplications modulo $U$ par exponentiation binaire.
	Ensuite, $W_{0}$ s’obtient sans calcul ; son degré est au plus égal à $n/p$. 
	Par conséquent, le temps de calcul $T(n)$ satisfait la récurrence $$T(n)\leq T\left(\frac{n}{p}\right)+C M(n)(log(n))$$ avec $C$ une constante.
\end{proof}

\section{Cas d'un polynôme sans facteur carré (Algorithme de Berlekamp)}
\label{Cas d'un polynôme sans facteur carré (Algorithme de Berlekamp)}

	On étudie d'abord l'algorithme de Berlekamp, car celui-ci est le plus générale.
	On suppose $P\in\mathbb{F}_{p}[X]$ un polynôme sans facteur carré de degré $n$, tel que $P=P_{1}\ldots P_{s}$ avec $P_{i}$ irréductibles.
	On a que $n=\sum_{1}^{s} i$.

\newtheorem*{theo3}{Théorème 4.2.1}
\begin{theo3}[Restes chinois]
	Soit $P=P_{1}\ldots P_{m}$ avec $P_{1},.\ldots, P_{m}$ $m$ polynômes irréductibles de $A=K[X]$ un corps deux à deux premiers entre eux.
	On pose $(P_{i})$ l'idéal engendré par $P_{i}$, alors $A/(P)$ est isomorphe à $A/(P_{1})\times\ldots\times A/(P_{m})$.
\end{theo3}

%\begin{proof}[Preuve]
%	A faire.
%\end{proof}

	D'après le théorème des restes chinois, on a $\mathbb{F}_{p}/(P)=\mathbb{F}_{p}/(P_{1})\times\ldots\times\mathbb{F}_{p}/(P_{s})$ avec $\mathbb{F}_{p}/(P_{i})\cong\mathbb{F}_{p^{deg(P_{i})}}$.

\newtheorem*{prop11}{Proposition 4.2.1}
\begin{prop11}
	Soit $N\in ker(M_{\varphi_{p}}-I_{p})$ un polynôme non constant.

	On a $$P=\prod\limits_{\substack{j\in\mathbb{F}_{p}}}{pgcd(P,N-j)}$$ et cette factorisation est non triviale.
\end{prop11}

\begin{proof}[Preuve]
	En remplaçant N dans l'équation suivante, $$X^{p}-X=\prod\limits_{\substack{j\in\mathbb{F}_{p}}}{(X-j)},$$ d'où $$N^{p}-N=\prod\limits_{\substack{j\in\mathbb{F}_{p}}}{(N-j)}.$$
	De plus, comme $P\mid X^{p}-X$, on a alors $$P=\prod\limits_{\substack{j\in\mathbb{F}_{p}}}{pgcd(P,N-j)}.$$
	On montre ensuite, par l'absurde, que la factorisation est non triviale.
	Si la factorisation est triviale, alors $\exists j$ telle que $P\mid N-j$, d'où $N$ constant modulo $P$.
	Ce qui absurde d'après l'énoncé.
\end{proof}


Voici donc l'algorithme de Berlekamp :

\begin{algorithm}[H]
\caption{Berlekamp}
\begin{algorithmic} 
\REQUIRE $P\in\mathbb{F}_{p}[X]$ un polynôme sans facteurs carré de degré $n$.
\ENSURE La factorisation de $P$.
\STATE On construit la matrice $M_{\varphi_{p}}$ engendrée par $\varphi_{p}$ l'endomorphisme de Frobenius dans la base $(1, x, \ldots, x^{n-1})$ avec $x$ l'image projective $X$ par la projection canonique;
\IF{$rg(M_{\varphi_{p}}-I_{p})\geq n$}
	\RETURN $P$ est irréductible;
\ELSE 
	\STATE On calcule le $ker(M_{\varphi_{p}}-I_{p})$ à l'aide d'un algorithme classique d'algèbre linéaire (par exemple un pivot de Gauss);
	\STATE On calcule le sous-espace propre associé à $1$ et on choisit des éléments de $N$ celui-ci;
	\STATE $i\gets 0$;
	\STATE $j\gets 1$;
	\STATE $F\gets 1$;
	\WHILE{$i\leq dim(ker(M_{\varphi_{p}}-I_{p}))$ \AND $j<p$}
		\IF{$pgcd(P,Q-j)$ premier à $F$} 
			\STATE $F\gets F\cdot pgcd(P,N-j)$;
			\STATE i++;
		\ENDIF
		\STATE j++;
	\ENDWHILE
\ENDIF 
\RETURN $F$;
\end{algorithmic}
\end{algorithm}

\newtheorem*{prop12}{Proposition 4.2.2}
\begin{prop12}
	L’algorithme a une complexité en $O(n^{\tau}+nM(n)log(n)p)$ avec $\tau$ dépendant du choix de l'algorithme de résolution de système linéaire.
\end{prop12}

\begin{proof}[Preuve]
	Tout d'abord, on commence par construire la matrice $M_{\varphi_{p}}$. 
	On commence par calculer $X^{p}\ mod\ P$ : soit $O(log(p))$ opérations $mod\ P$, soit $O(M(n) log(p))$ opérations dans $\mathbb{F}_{p}$.
	Ensuite, il faut calculer $(X^{2})^{p}=(X^{p})^{2}, (X^{3})^{p}=(X^{p})^{3}, \ldots,$ ce qui se fait en $n$ multiplications mod $P$, soit $O(nM(n))$ opérations en tout.

	La résolution du système linéaire d'un algorithme l'algèbre linéaire (tel un pivot de Gauss) coûte $n^{\tau}$.

	Une fois obtenu un vecteur du noyau, pour chaque essai de $pgcd$ un coûte $O(M(n)log(n))$ opérations avec un maximum de de $p$.

	Pour trouver la factorisation complété, il suffit de réitérer le procédé ce qui entraîne a priori un surcoût de n.
	En fait, il n’y a pas besoin de refaire de l’algèbre linéaire, de sorte que le coût total de la factorisation est $O(n^{\tau} + nM(n) log(n) p)$.
\end{proof}

\section{Application de l'Algorithme de Berlekamp}
\label{Application de l'Algorithme de Berlekamp}

	Dans cette section, on étudie une application de l'algorithme de Berlekamp sur un exemple concret.

	On choisi par exemple le polynôme $$P=X^{6}+7\in\mathbb{F}_{11}$$ et on cherche à le factoriser sur $\mathbb{F}_{11}$.
	On note $x$ l'image projective $X$ par la projection canonique de $\mathbb{F}_{11}$ dans $\mathbb{F}_{11}/(P)$ et 
$$\begin{array}{clcl}
	\varphi_{11} : &\mathbb{F}_{11}[X] &\longrightarrow &\mathbb{F}_{11}[X]\\
								 & t 								&\longmapsto			&t^{11}\\
\end{array}.$$
	On a :
$$\left\{
\begin{array}{r c l}
\varphi_{11}(1) &\equiv& 1\ [P]\\
\varphi_{11}(x) &\equiv& 4x^{5}\ [P]\\
\varphi_{11}(x^{2}) &\equiv& -2x^{4}\ [P]\\
\varphi_{11}(x^{3}) &\equiv& x^{3}\ [P]\\
\varphi_{11}(x^{4}) &\equiv& 5x^{2}\ [P]\\
\varphi_{11}(x^{5}) &\equiv& 3x\ [P]\\
\end{array}$$
	On peux ensuite construire la matrice associée à $\varphi_{11}$, on obtient ainsi 
$$M_{\varphi_{11}}=
\begin{pmatrix}
1&0&0&0&0&0\\
0&0&0&0&0&3\\
0&0&0&0&5&0\\
0&0&0&1&0&0\\
0&0&-2&0&0&0\\
0&4&0&0&0&0\\
\end{pmatrix}$$
et
$$M_{\varphi_{11}}-I_{11}=
\begin{pmatrix}
0&0&0&0&0&0\\
0&-1&0&0&0&3\\
0&0&-1&0&5&0\\
0&0&0&0&0&0\\
0&0&-2&0&-1&0\\
0&4&0&0&0&-1\\
\end{pmatrix}.$$
	On peux remarquer que $rg(M_{\varphi_{11}}-I_{11})=2$, par conséquent il y a quatre facteurs irréductibles et le noyau de $ker(M_{\varphi_{11}}-I_{11})=\left\langle 1, x^{3}, 5x^{2}+x^{4}, 3x+x^{5}\right\rangle$ (grâce à un algorithme d'algèbre linéaire). On a :
$$pgcd(X^{6}+7,3XX^{5}+1)=X^{2}+4X+5,\ pgcd(X^{6}+7,3X+X^{5}+2)=X+4,$$
$$pgcd(X^{6}+7,3X+X^{5}+9)=X+7,\ pgcd(X^{6}+7,3X+X^{5}+10)=X^{2}+7X+5.$$
	On obtient donc une factorisation de $P$ modulo $11$, d'où $$P=(X^{2}+7X+5)(X^{2}+4X+5)(X+7)(X+4).$$

%\section{Algorithme de Cantor-Zassenhaus}
%\label{Algorithme de Cantor-Zassenhaus}
%
%	A compléter...