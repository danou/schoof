\documentclass{article}% option leqno pour aligner les numéros d'équations à gauche
                                    % option fleqn pour aligne les équations à gauche

\usepackage[utf8]{inputenc}%          encodage en utf8 du fichier soruce
\usepackage[T1]{fontenc}%             encodage de police pour gérer les accents français
\usepackage[francais]{babel}%         documen en langue française uniquement
\usepackage{textcomp}%                caractères additionnels
\usepackage{amsmath,amssymb,amsthm}%  packages de l'AMS
\usepackage{txfonts}%                 pour utilsier la police times
\usepackage[a4paper]{geometry}%       pour gérer la taille du papier et les marges
\usepackage{graphicx}%                pour inclure des images
\usepackage{xcolor}%                  pour gérer les couleurs
\usepackage{microtype}%               diverses améliorations typographiques

\usepackage[Algorithme]{algorithm}
\usepackage{algorithmic}

\usepackage{wrapfig}%                 pour mettre des images dans le texte

\usepackage{titling}%                 pour personnaliser le titre
\usepackage[runin]{abstract}%         pour personnaliser l'abstract ; l'option runin ne met pas le titre au-dessus
%\usepackage{titlesec}%                pour personnaliser les sections
\usepackage{titletoc}%                pour personnaliser la table des matières
\usepackage{fancyhdr}%                pour personnaliser les en-têtes et pieds de pages
\usepackage{footmisc}%                pour personnaliser les footnotes

\usepackage{hyperref}%                gestion des hyperliens --> À METTRE EN DERNIER
\hypersetup{pdfstartview=XYZ}%        restauration du zoom par défaut


% PERSONNALISATION DE LA PAGE
% marges
\geometry{top=3cm,bottom=2.5cm}


% PERSONNALISATION DU TITRE AVEC TITLING
% au-dessus du titre
\setlength{\droptitle}{-1cm}
\renewcommand{\maketitlehooka}{{\setlength{\parindent}{0cm}\scriptsize
  \emph{Master 2 Algèbre Appliquées} \hfill

  Université Paris-Saclay \hfill Reproduction interdite
  
  2016 - 2017 \hfill Tous droits réservés
  
  \vspace{1cm}
}}
% titre
\pretitle{\begin{center}\large\bfseries\MakeUppercase}
\posttitle{\par\rule{5cm}{0.56pt}\end{center}}
% auteur
\preauthor{\begin{center}
  \textit{par} \scshape \lineskip 0.5em%
}
\postauthor{\par\end{center}}
\renewcommand{\and}{\unskip, }% parce qu'on a supprimé le tableau
% date
\predate{\begin{center}\small le }
%\postdate{\par\end{center}\vspace{1cm}}


% PERSONNALISATION DU RÉSUMÉ AVEC LE PACKAGE ABSTRACT
\renewcommand{\abstractnamefont}{\normalfont\small\scshape}
\abslabeldelim{. ---}
\setlength{\absleftindent}{\parindent}
\setlength{\absrightindent}{\parindent}
\setlength{\abstitleskip}{-\parindent}


% PERSONNALISATION DES THÉORÈMES
% théorèmes
\newtheoremstyle{plain}
  {\topsep}%   espacement avant le théorème
  {\topsep}%   espacement après le théorème
  {\itshape}%  police du corps du théorème
  {}%          indentation (vide pour aucune indentation, sinon \parindent ou une autre longueur)
  {\scshape}%  police du titre du théorème  
  {}%     ponctuation après le titre du théorème
  {\newline}%         espace après le titre du théorème (soit une espace, soit une longueur soit un \newline)
  {\llap{}\thmname{#1}\thmnumber{ #2}\thmnote{ \normalfont(#3)}}% spécification du titre
\theoremstyle{plain}
\newtheorem{theoreme}{Théorème}[section]
\renewcommand{\thetheoreme}{\thesection.\textsc{\roman{theoreme}}}

% définition
\theoremstyle{definition}
\newtheorem{definition}{Définition}[section]


% proposition
\theoremstyle{plain}
\newtheorem{proposition}{Proposition}[section]

% remarque
\newtheoremstyle{remark}
  {\topsep}%   space before
  {\topsep}%   space after
  {}%  Body font
  {}%          Indent amount (empty for no indent, \parindent)
  {\itshape}% Thm head font   
  {. ---}%         Punctuation after thm head
  { }%         Space after thm head (\newline = linebreak)
  {\thmname{#1}\thmnumber{ #2}\thmnote{ \normalfont(#3)}}% Thm head spec
\theoremstyle{remark}
\newtheorem*{remarque}{Remarque}





% PERSONNALISATION DE LA TABLE DES MATIÈRES AVEC LE PACKAGE TITLETOC
\addto\captionsfrench{\renewcommand{\contentsname}{Sommaire}}
%\contentsmargin{2em}% il faut augmenter la largeur de \contentspage car les numéros de pages ont 3 chiffres
\titlecontents{section}
  [0pc]
  {}
  {§~\thecontentslabel.~}
  {}
  {\titlerule*[0.75em]{.}\contentspage}
 \titlecontents{subsection}
  [0pc]
  {}
  {§~\thecontentslabel.~}
  {}
  {\titlerule*[0.75em]{.}\contentspage}


% PERSONNALISATION DES NUMÉROS DE PAGE
%\setcounter{page}{356}


% PERSONNALISATION DES EN-TÊTES AVEC LE PACKAGE FANCYHDR
\pagestyle{fancy}
\fancyhead{}
\makeatletter
\fancyhead[R]{\itshape\@title}
\makeatother
\fancyhead[L]{\scshape\theauthor}
\renewcommand{\headrulewidth}{0pt}


% PERSONNALISATION DES NUMÉROS D'ÉQUATION
%\numberwithin{equation}{section}
%\renewcommand{\theequation}{\thesection.\textit{\alph{equation}}}


% PERSONNALISATION DES FOOTNOTE AVEC LE PACKAGE FOOTMISC
% numérotation en \fnsymbol des footnotes
\renewcommand{\thefootnote}{\fnsymbol{footnote}}
% utilisation de footmisc pour redéfinir \fnsymbol
\DefineFNsymbols*{daggers}{%
  {\textdagger}%
  {\textdaggerdbl}%
  {\textdagger\textdagger}%
  {\textdaggerdbl\textdaggerdbl}%
  {\textdagger\textdagger\textdagger}%
  {\textdaggerdbl\textdaggerdbl\textdaggerdbl}%
  {\textdagger\textdagger\textdagger\textdagger}%
  {\textdaggerdbl\textdaggerdbl\textdaggerdbl\textdaggerdbl}%
}
\setfnsymbol{daggers}
% désactivation de babel pour l'apparence de la footnote en bas de la page
\frenchbsetup{FrenchFootnotes=false}


% TITRE DU DOCUMENT
\title{Comptage de points de courbes elliptique sur des corps finis}
\author{Daniel RESENDE}
\date{\today}

\newcommand\fq{\mathbf{F}_{q}}
\newcommand\ie{\textit{i.e.}}

\begin{document}

\maketitle

\begin{abstract}
Il s'agit de la description de l'algorithme de René Schoof. Celui-ci fût le premier algorithme de comptage de points de courbes elliptique sur des corps finis en un temps polynomial ($O(\log^{9} p)$).  
\end{abstract}

\tableofcontents

\clearpage\addcontentsline{toc}{section}{Introduction}
\section*{Introduction}

\begin{wrapfigure}{r}{0cm}
\includegraphics[width=3cm]{Rene_Schoof.jpg}
\caption{\label{étiquette}Portrait René Schoof}
\end{wrapfigure}
Les courbes elliptiques définissent une lois de groupe sur les corps finis $\fq$ qui est difficile pour le problème du logarithme discret. On retrouve par conséquent son utilisation dans plusieurs schémas cryptographiques comme Diffie-Hellman (avec ECDH) ou El-Gamal (avec ECDSA).
Cependant, l'utilisation de schémas à l'aide de courbes elliptiques nécessite d'avoir un grand nombre premier qui divise l'ordre d'un sous-groupe cyclique de la courbe de $E(\fq)$. Nous avons donc besoin de connaitre le cardinal de $E(\fq)$.

\begin{list}{•}{Il existe aujourd'hui de nombres algorithme de comptage de points d'une courbes elliptiques sur un corps finis $\fq$ :}
\item L'algorithme Baby Step Giant Step basé sur le théorème de Hasse,
\item L'algorithme Schoof en 1985 que l'on va étudier dans ce mémoire,
\item L'algorithme SEA [Schoof, Elkies, Atkin] en 1995 qui est une amélioration de l'algorithme de Schoof,
\item L'algorithme de Satoh en 2005 basé sur le relèvement canonique sur les $\mathbf{Z}$ $q$-adiques,
\item L'algorithme AGM [Mestre] basé sur le calcul de suites arithmetico-géométriques.
\end{list}

Dans ce projet, je vais vous présenter un algorithme de comptage de points de courbes elliptique sur des corps finis. Je me restreindrais à des corps finis $\fq$ avec $q=p^{n}$ et $p$ premier différent de 2 et 3. Pour c'est deux derniers cas, l'algorithme est sensiblement le même. 

%\clearpage
\section{Courbes elliptiques sur $\fq$}

Soit $\fq$ un corps fini à $p$ éléments de caractéristiques $p\neq 2,3$.

Soit $E$ une courbe elliptique définie sur $\fq$. On obtient l'équation affine de Weierstra\ss : 
\begin{equation}
y^{2} = x^{3} + ax + b
\end{equation} 
avec $a,b\in\fq$ et $\Delta = -16(4a^{3} + 27b^{2}) \neq 0$.

\begin{definition}
Soit $\varPhi$ l'endomorphisme de Frobénius d'une courbe elliptique $E$ tel que  
$$\begin{array}{clcl}
\varPhi : &E(\bar{\fq}) &\longrightarrow &E(\bar{\fq})\\
&(x, y) &\longmapsto	&(x^{p}, y^{p}).\\
\end{array}$$
\end{definition}

\begin{definition}[Trace]
Soit $E$ une courbe elliptique sur $\fq$. La trace de $E(\fq)$ est l'entier $t=q+1-\#E(\fq)$.
\end{definition}

\begin{proposition}
Soit la trace $t$ de $E(\fq)$, on a alors 
\begin{equation}
\phi^{2} - t\phi + q = 0
\end{equation}
\end{proposition}

\begin{theoreme}[de Hasse]
Soit $E$ une courbe elliptique sur $\fq$ et la trace $t$ de $E(\fq)$.
On a \begin{equation}
\mid t\mid\leq 2\sqrt{q},
\end{equation}
et par conséquent
\begin{equation}
\mid\#E(\fq)-(q+1)\mid\leq 2\sqrt{q}
\end{equation}
\end{theoreme}

Nous allons maintenant nous concentrer sur les sous-groupe de $n$-torsions $E[n]$ avec $n\in\mathbf{Z}_{\geq -1}$ tel que $p\nshortmid n$. Et on introduit la notion de polynôme de division.

\begin{definition}[Polynôme de division]
Soit $n\in\mathbb{Z}^{*}$, le polynôme de division $\psi_{n}$ est la fonction polynôme de $K[E]$ de coefficient dominent $n$ et de diviseur $$div(\psi_{n})=(E[n])-n^{2}(\vartheta)$$
\end{definition}

\begin{proposition}[Caractérisation du polynôme de division]
On construit le polynôme de division par récurrence sur $n\in\mathbf{Z}_{\geq 1}$ :
\begin{enumerate}
\item $\psi_{-1}(X,Y)=-1,\ \psi_{0}(X,Y)=0,\ \psi_{1}(X,Y)=1,\ \psi_{2}(X,Y)=2Y$,
\item $\psi_{3}(X,Y)=3X^{4} + 6aX^{2} + 12bX - a^{2}$,
\item $\psi_{4}(X,Y)=4Y(X^{6} + 5aX^{4} + 20bX^{3} - 5a^{2}X^{2} - 4bX - 8b^{2} - a^{3}$,
\item $\psi_{2n}(X,Y)=\psi_{n}(\psi_{n+2}\psi_{n-1}^{2} - \psi_{n-2}\psi_{n+1}^{2})/2Y$,
\item $\psi_{2n+1}(X,Y)=\psi_{n+2}\psi_{n}^{3} - \psi_{n+1}^{3}\psi_{n-1}$,
\item $\psi_{-n}=\psi_{n}$.
\end{enumerate}
\end{proposition}

\begin{proof}
Voir \cite{ref4}.
\end{proof}

Dans l'algorithme de Schoof, nous utiliserons une variante du polynôme de division.

\begin{definition}
Soit $n\in\mathbb{Z}^{*}$, le polynôme $f_{n}$ est une fonction polynôme de $K[E]$ définie par les relations suivantes : 
$$
f(n) = \left\{
\begin{array}{l l}
  \bar{\psi_{n}}(X,Y) & \quad \text{si } n \text{ est pair}\\
  \bar{\psi_{n}}(X,Y)/Y & \quad \text{si } n \text{ est impair}\\ \end{array} \right.
$$
où $\bar{\psi_{n}}$ est la réduction de $\psi_{n}$ par les termes en $Y^{2}$ par l'équation $(E)$.
\end{definition}

\begin{proposition}[Caractérisation de $f_{n}$]
On construit $f_{n}$ par récurrence sur $n\in\mathbf{Z}_{\geq 1}$ :
\begin{enumerate}
\item $f_{-1}(X)=-1,\ f_{0}(X)=0,\ f_{1}(X)=1,\ f_{2}(X)=2$,
\item $\psi_{3}(X)=3X^{4} + 6aX^{2} + 12bX - a^{2}$,
\item $\psi_{4}(X)=4Y(X^{6} + 5aX^{4} + 20bX^{3} - 5a^{2}X^{2} - 4bX - 8b^{2} - a^{3}$,
\item $f_{2n}(X,Y)=f_{n}(f_{n+2}f_{n-1}^{2} - f_{n-2}f_{n+1}^{2})$,

\item $$
f(n) = \left\{ 
\begin{array}{l l}
  \bar{\psi_{n}}(X,Y) & \quad \text{si } n \text{ est pair}\\
  \bar{\psi_{n}}(X,Y)/Y & \quad \text{si } n \text{ est impair}\\ \end{array} \right.
$$


\item $f_{2n+1}(X,Y)=\psi_{n+2}\psi_{n}^{3} - \psi_{n+1}^{3}\psi_{n-1}$,
\end{enumerate}
\end{proposition}

\begin{proof}
Voir \cite{ref4}.
\end{proof}






\clearpage
\section{Algorithme de Schoof}
\subsection{Cas général}

Cette algorithme consiste à calculer la trace du frobénius modulo tous les $l<l_{max}$ tel que $l_{max}$ soit le plus grand nombre premier vérifiant :\begin{equation} 
\prod_{l\ premier,\ p\nshortmid l}^{l_{max}}l > 4\sqrt{q}.
\end{equation}
Une fois calculé la trace modulo toutes les $l$-torsions, on utilise le Théorème des Restes Chinois (CRT) pour obtenir la trace dans $\fq$. Puis on utilise le théoreme de Hasse pour avoir le cardinal de la courbe $E$ sur $\fq$.

\begin{theoreme}[Algorithme de Schoof]
Voici le descriptif de l’algorithme de Schoof:

\begin{algorithm}
\caption{Algorithme de Shoof}
\label{schoof1}
\begin{algorithmic} 
\REQUIRE Une courbe elliptique $E$ sur $\fq$ un polynôme quelconque.
\ENSURE Le cardinal de $E(\fq)$.
\STATE $M\leftarrow 2, l\leftarrow 3$;
\STATE $S\leftarrow \{(t\ mod\ 2, 2)\}$; \COMMENT{Cas pour $l = 2$}
\WHILE{$M < 4\sqrt{q}$}
    \STATE $k\leftarrow q\ mod\ l$;	
    \FOR{$\tau = 0$ \TO $\frac{l - 1}{2}$}
        \IF{$\forall P\in E[l],\ \phi^{2}(P) + [k]P = \pm[\tau]\phi(P)$}
            \STATE $S\leftarrow S\cup \{(\tau, l)\}$ \OR $S\leftarrow S\cup \{(-\tau, l)\}$ \COMMENT{Selon les cas}
            \STATE break;
        \ENDIF
    \ENDFOR
    \STATE $M\leftarrow M*l$;
    \STATE $l\leftarrow\ nextprime(l)$; \COMMENT{Donne le prochain nombre premier après $l$}	
\ENDWHILE
\STATE $\forall t\in S,\ trace\leftarrow CRT(t)$; \COMMENT{Effectue le théorème des restes chinois}
\RETURN $q + 1 - trace$.
\end{algorithmic}
\end{algorithm}
\end{theoreme}

\begin{proof}
\begin{description}
\item
\item[Cas mod 2] Dans ce cas, on cherche les points de 2-torsions, \ie les points spéciaux de la courbes.
$$t=1\ mod\ 2 \Leftrightarrow \#E(\fq)[2] = 1 \Leftrightarrow X^{3} + aX + b\text{ est irreductible sur }\fq\Leftrightarrow pgcd(X^{3} + aX + b, X^{q} - X) = 1$$
\item 
\end{description}
\end{proof}










\subsection{Amélioration de Schoof}

Dans son article original, Schoof (voir \cite{ref1}) propose une amélioration possible de son algorithme.
\begin{itemize}
\item Si $\forall P$ nonzéro $\phi_{l}^{2}P = \pm kP$ avec $q\equiv k[l]$ 
\item Sinon on fait le cas général.
\end{itemize}

\begin{algorithm}
\caption{Algorithme de Shoof amélioré}
\label{schoof2}
\begin{algorithmic} 
\REQUIRE Une courbe elliptique $E$ sur $\fq$ un polynôme quelconque.
\ENSURE Le cardinal de $E(\mathbf{F}_{p})$.
\STATE $M\leftarrow 2, l\leftarrow 3$;
\STATE $S\leftarrow \{(t\ mod\ 2, 2)\}$; \COMMENT{Cas pour $l = 2$}
\WHILE{$M < 4\sqrt{q}$}
    \STATE $k\leftarrow q\ mod\ l$;	
    \IF{$\phi_{l}^{2}P = \pm kP$}	
        \IF{$(\frac{k}{l}) = -1$}
            \STATE $S\leftarrow S\cup \{(0, l)\}$
            \ELSE
                \STATE on recherche $w$ tel que $k=w^{2}\ mod\ l$
                \IF{$\pm w$ est une valeur propre de $\phi_{l}$}
                    \STATE $S\leftarrow S\cup \{(w, l)\}$ \OR $S\leftarrow S\cup \{(-w, l)\}$ \COMMENT{Selon les cas}
                \ELSE
                    \STATE $S\leftarrow S\cup \{(0, l)\}$
                \ENDIF
            \ENDIF
    \ELSE
        \FOR{$\tau = 0$ \TO $\frac{l - 1}{2}$}
            \IF{$\forall P\in E[l],\ \phi^{2}(P) + [k]P = \pm[\tau]\phi(P)$}
                \STATE $S\leftarrow S\cup \{(\tau, l)\}$ \OR $S\leftarrow S\cup \{(-\tau, l)\}$ \COMMENT{Selon les cas}
                \STATE break;
            \ENDIF
        \ENDFOR
    \ENDIF
    \STATE $M\leftarrow M*l$;
    \STATE $l\leftarrow\ nextprime(l)$; \COMMENT{Donne le prochain nombre premier après $l$}	
\ENDWHILE
\STATE $\forall t\in S,\ trace\leftarrow CRT(t)$; \COMMENT{Effectue le théorème des restes chinois}
\RETURN $q + 1 - trace$.
\end{algorithmic}
\end{algorithm}

\clearpage 

\section{Étude de la complexité}




 
\clearpage 
\nocite{*} 
\bibliographystyle{alpha}
\bibliography{Bibliographie}
\end{document}