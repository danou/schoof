\documentclass[fleqn,leqno]{article}% option leqno pour aligner les numéros d'équations à gauche
                                    % option fleqn pour aligne les équations à gauche

\usepackage[utf8]{inputenc}%          encodage en utf8 du fichier soruce
\usepackage[T1]{fontenc}%             encodage de police pour gérer les accents français
\usepackage[francais]{babel}%         documen en langue française uniquement
\usepackage{textcomp}%                caractères additionnels
\usepackage{amsmath,amssymb,amsthm}%  packages de l'AMS
\usepackage{txfonts}%                 pour utilsier la police times
\usepackage[a4paper]{geometry}%       pour gérer la taille du papier et les marges
\usepackage{graphicx}%                pour inclure des images
\usepackage{xcolor}%                  pour gérer les couleurs
\usepackage{microtype}%               diverses améliorations typographiques

\usepackage[Algorithme]{algorithm}
\usepackage{algorithmic}

\usepackage{wrapfig}%                 pour mettre des images dans le texte

\usepackage{titling}%                 pour personnaliser le titre
\usepackage[runin]{abstract}%         pour personnaliser l'abstract ; l'option runin ne met pas le titre au-dessus
\usepackage{titlesec}%                pour personnaliser les sections
\usepackage{titletoc}%                pour personnaliser la table des matières
\usepackage{fancyhdr}%                pour personnaliser les en-têtes et pieds de pages
\usepackage{footmisc}%                pour personnaliser les footnotes

\usepackage{hyperref}%                gestion des hyperliens --> À METTRE EN DERNIER
\hypersetup{pdfstartview=XYZ}%        restauration du zoom par défaut


% PERSONNALISATION DE LA PAGE
% marges
\geometry{top=3cm,bottom=2.5cm}


% PERSONNALISATION DU TITRE AVEC TITLING
% au-dessus du titre
\setlength{\droptitle}{-1cm}
\renewcommand{\maketitlehooka}{{\setlength{\parindent}{0cm}\scriptsize
  \emph{Master 2 Algèbre Appliquées} \hfill

  Université Paris-Saclay \hfill Reproduction interdite
  
  2016 - 2017 \hfill Tous droits réservés
  
  \vspace{1cm}
}}
% titre
\pretitle{\begin{center}\large\bfseries\MakeUppercase}
\posttitle{\par\rule{5cm}{0.56pt}\end{center}}
% auteur
\preauthor{\begin{center}
  \textit{par} \scshape \lineskip 0.5em%
}
\postauthor{\par\end{center}}
\renewcommand{\and}{\unskip, }% parce qu'on a supprimé le tableau
% date
\predate{\begin{center}\small le }
%\postdate{\par\end{center}\vspace{1cm}}


% PERSONNALISATION DU RÉSUMÉ AVEC LE PACKAGE ABSTRACT
\renewcommand{\abstractnamefont}{\normalfont\small\scshape}
\abslabeldelim{. ---}
\setlength{\absleftindent}{\parindent}
\setlength{\absrightindent}{\parindent}
\setlength{\abstitleskip}{-\parindent}


% PERSONNALISATION DES THÉORÈMES
% théorèmes
\newtheoremstyle{plain}
  {\topsep}%   espacement avant le théorème
  {\topsep}%   espacement après le théorème
  {\itshape}%  police du corps du théorème
  {}%          indentation (vide pour aucune indentation, sinon \parindent ou une autre longueur)
  {\scshape}%  police du titre du théorème  
  {. ---}%     ponctuation après le titre du théorème
  { }%         espace après le titre du théorème (soit une espace, soit une longueur soit un \newline)
  {\llap{}\thmname{#1}\thmnumber{ #2}\thmnote{ \normalfont(#3)}}% spécification du titre
\theoremstyle{plain}
\newtheorem{theoreme}{Théorème}[section]
\renewcommand{\thetheoreme}{\thesection.\textsc{\roman{theoreme}}}

% définition
\newtheoremstyle{definition}
  {\topsep}%   espacement avant la définition
  {\topsep}%   espacement après la défintion
  {\itshape}%  police du corps du théorème
  {}%          indentation (vide pour aucune indentation, sinon \parindent ou une autre longueur)
  {\scshape}%  police du titre du théorème  
  {. ---}%     ponctuation après le titre du théorème
  { }%         espace après le titre du théorème (soit une espace, soit une longueur soit un \newline)
  {\llap{}\thmname{#1}\thmnumber{ #2}\thmnote{ \normalfont(#3)}}% spécification du titre
\theoremstyle{definition}
\newtheorem{definition}{Définition}[section]
\renewcommand{\thedefinition}{\thesection.\textsc{\roman{definition}}}

% remarque
\newtheoremstyle{remark}
  {\topsep}%   space before
  {\topsep}%   space after
  {}%  Body font
  {}%          Indent amount (empty for no indent, \parindent)
  {\itshape}% Thm head font   
  {. ---}%         Punctuation after thm head
  { }%         Space after thm head (\newline = linebreak)
  {\thmname{#1}\thmnumber{ #2}\thmnote{ \normalfont(#3)}}% Thm head spec
\theoremstyle{remark}
\newtheorem*{remarque}{Remarque}


% PERSONNALISATION DES SECTIONS AVEC LE PACKAGE TITLESEC
\renewcommand{\thesection}{\oldstylenums{\arabic{section}}}
\titleformat{\section}[hang]
  {\normalfont\large\bfseries\scshape}
  {§~\thesection.~}
  {0em}
  {}


% PERSONNALISATION DE LA TABLE DES MATIÈRES AVEC LE PACKAGE TITLETOC
\addto\captionsfrench{\renewcommand{\contentsname}{Sommaire}}
%\contentsmargin{2em}% il faut augmenter la largeur de \contentspage car les numéros de pages ont 3 chiffres
\titlecontents{section}
  [0pc]
  {}
  {§~\thecontentslabel.~}
  {}
  {\titlerule*[0.75em]{.}\contentspage}


% PERSONNALISATION DES NUMÉROS DE PAGE
%\setcounter{page}{356}


% PERSONNALISATION DES EN-TÊTES AVEC LE PACKAGE FANCYHDR
\pagestyle{fancy}
\fancyhead{}
\makeatletter
\fancyhead[R]{\itshape\@title}
\makeatother
\fancyhead[L]{\scshape\theauthor}
\renewcommand{\headrulewidth}{0pt}


% PERSONNALISATION DES NUMÉROS D'ÉQUATION
\numberwithin{equation}{section}
\renewcommand{\theequation}{\thesection.\textit{\alph{equation}}}


% PERSONNALISATION DES FOOTNOTE AVEC LE PACKAGE FOOTMISC
% numérotation en \fnsymbol des footnotes
\renewcommand{\thefootnote}{\fnsymbol{footnote}}
% utilisation de footmisc pour redéfinir \fnsymbol
\DefineFNsymbols*{daggers}{%
  {\textdagger}%
  {\textdaggerdbl}%
  {\textdagger\textdagger}%
  {\textdaggerdbl\textdaggerdbl}%
  {\textdagger\textdagger\textdagger}%
  {\textdaggerdbl\textdaggerdbl\textdaggerdbl}%
  {\textdagger\textdagger\textdagger\textdagger}%
  {\textdaggerdbl\textdaggerdbl\textdaggerdbl\textdaggerdbl}%
}
\setfnsymbol{daggers}
% désactivation de babel pour l'apparence de la footnote en bas de la page
\frenchbsetup{FrenchFootnotes=false}


% TITRE DU DOCUMENT
\title{Comptage de points de courbes elliptique sur des corps finis}
\author{Daniel RESENDE}
\date{\today}

\begin{document}

\maketitle

\begin{abstract}
Il s'agit de la description de l'algorithme de René Schoof. Celui-ci fût le premier algorithme de comptage de points de courbes elliptique sur des corps finis en un temps polynomial ($O(\log^{9} p)$).  
\end{abstract}

\begin{remarque}
Les éléments biographiques sont tirés de ..
\end{remarque}

\tableofcontents

\clearpage\addcontentsline{toc}{section}{Introduction}
\section*{Introduction}

Dans ce projet, je vais vous présenter un algorithme de comptage de points de courbes elliptique sur des corps finis. Je me restreindrais à des corps finis $\mathbf{F}_{p}$ avec $p$ premier différent de 2 et 3. Pour c'est deux derniers cas, l'algorithme est sensiblement le même. 

\subsection{Contexte historique}

%\begin{wrapfigure}{r}{0cm}\includegraphics[width=3cm]{200px-Leonhard_Euler_2}\end{wrapfigure}%



\section{Courbes elliptiques sur $\mathbf{F}_{p}$}

Soit $\mathbf{F}_{p}$ un corps fini à $p$ éléments de caractéristiques $p\neq 2,3$.

Soit $E$ une courbe elliptique définie sur $\mathbf{F}_{p}$. On obtient l'équation affine de Weierstra\ss : $$y^{2} = x^{3} + ax + b$$ avec $a,b\in\mathbf{F}_{p}$ et $\Delta = -16(4a^{3} + 27b^{2}) \neq 0$.

\begin{definition}
Soit $\varPhi$ l'endomorphisme de Frobénius d'une courbe elliptique $E$ tel que  
$$\begin{array}{clcl}
\varPhi : &E(\bar{\mathbf{F}_{p}}) &\longrightarrow &E(\bar{\mathbf{F}_{p}})\\
&(x, y) &\longmapsto	&(x^{p}, y^{p}).\\
\end{array}$$
\end{definition}

\section{Algorithme de Schoof}

\subsection{Cas général}

\begin{algorithm}
\caption{Algorithme de Shoof}
\label{schoof}
\begin{algorithmic} 
\REQUIRE Une courbe elliptique $E$ sur $\mathbf{F}_{p}$ un polynôme quelconque.
\ENSURE Le cardinal de $E(\mathbf{F}_{p})$.
\STATE $M\leftarrow 2, l\leftarrow 3$;
\STATE $S\leftarrow \{(t\ mod\ 2, 2)\}$; \COMMENT{Cas pour $l = 2$}
\WHILE{$M < 4\sqrt{q}$}
	\STATE $k\leftarrow q\ mod\ l$;	
	\FOR{$\tau = 0$ \TO $\frac{l - 1}{2}$}
		\IF{$\forall P\in E[l],\ \varphi^{2}(P) + [k]P = \pm[\tau]\varphi(P)$}
			\STATE $S\leftarrow S\cup \{(\tau, l)\}$ \OR $S\leftarrow S\cup \{(-\tau, l)\}$ \COMMENT{Selon les cas}
			\STATE break;
		\ENDIF
	\ENDFOR
	\STATE $M\leftarrow M*l$;
	\STATE $l\leftarrow\ nextprime(l)$; \COMMENT{Donne le prochain nombre premier après $l$}	
\ENDWHILE
\STATE $\forall t\in S,\ trace\leftarrow CRT(t)$; \COMMENT{Effectue le théorème des restes chinois}
\RETURN $q + 1 - trace$.
\end{algorithmic}
\end{algorithm}

\subsection{Amélioration de Schoof}

Dans son article original, Schoof (voir \cite{ref1}) propose une amélioration possible de son algorithme.
\begin{itemize}
\item Si $\forall P$ nonzéro $\phi_{l}^{2}P = \pm kP$ avec $q\equiv k[l]$ 
\item Sinon on fait le cas général.
\end{itemize}




\section{Quelques résultats mathématiques marquants}

Une des réussites d'Euler a été la démonstration du grand théorème de Fermat dans un cas particulier\footnote{Fermat lui-même n'avait de démonstration que dans le cas $n=4$.}.

\begin{theoreme}[de Fermat, cas $n=3$]\label{theoreme.I}
L'équation $x^3 + y^3 + z^3 = 0$ n'admet aucune solutions entières lorsque $xyz \neq 0$.
\end{theoreme}

\begin{proof}
On renvoie à \cite{ref1}
\end{proof}

Le théorème~\ref{theoreme.I} est un résultat de théorie des nombres, mais Euler a touché à d'autres domaines. Citons par exemple ce résultat de topologie.

\begin{theoreme}
Il n'est pas possible de traverser tous les ponts de Könisberg en ne passant qu'une seule fois sur chaque pont.
\end{theoreme}


 
\nocite{*}

 
\bibliographystyle{alpha}
\bibliography{Bibliographie}


\end{document}
